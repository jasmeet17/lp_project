\documentclass [12pt] {article}
\usepackage{amsmath}
\makeatletter
\renewcommand{\@seccntformat}[1]{}
\makeatother
\usepackage{url}
\usepackage[margin=0.8in]{geometry}
\pagestyle{plain}
\begin{document}
\section*{Initial Matrices and Vector}
\[
A  =
\begin{bmatrix}
1 & -1 & 1 & 0 & 0 \\
2 & -1  & 0 &1 & 0\\
0 & 1 & 0 & 0 & 1\\
\end{bmatrix}
\]
\\
\[ Initial\ set\  of\  basic\  and\  nonbasic\  indices \]
\[
\beta= \left\{3,4,5\right\} \quad and \quad  \nu=\left\{1,2\right\}
\]
\\
\[
Submatrice\ of \ A
\]

\[
B =
\begin{bmatrix}
1 & 0 & 0 \\
0 &1 & 0\\
0 & 0 & 1\\
\end{bmatrix} \quad and \quad
\mathit{N} =
\begin{bmatrix}
1 & -1 \\
2 &-1\\
0 &1\\
\end{bmatrix}
\]
\\
\[
Inital\ values\ of\ the\ basic\ variables\ are\ given\ by
\]
\[
x_B^* = b =
\begin{bmatrix}
1 \\
2 \\
0\\
\end{bmatrix}
\]
\[
Inital\ values\ of\ the\ nonbasic\ dual variables\ are\ given\ by\
\]
\[
z_\mathit{N}^*= -c_\mathit{N} =
\begin{bmatrix}
-4 \\
-3 \\
\end{bmatrix}
\]
\\
\[
Since\ x_B^*\  \geq \  0, the\ initial\ solution\ is\ primal\ feasible.
\]
\section*{First Iteration}
\subsection{Step 1.}
\[
Since\ z_\mathit{N}^*\ has\ some\ negative\ components,\ the\ current\ solution\ is\ not\ optimal.\
\]

\subsection{Step 2.}
\[
Since\ z_\mathit{N}^*\ = \ -4\ and\ this\ is\ the\ most\ negative\ dual\ variables,
\]

\[
we\ see\ that\ the\ entering\ index\ is\  j\ =\ 1
\]


\subsection{Step 3.}

\[
\Delta x_{\mathcal B} = B^{-1} N e_j =
\begin{bmatrix}
1 & -1 \\
2 & -1 \\
0 & 1 \\
\end{bmatrix}
\begin{bmatrix}
1 \\
0\\
\end{bmatrix}
= \begin{bmatrix}
1 \\
2 \\
0\\
\end{bmatrix}
\]
\subsection{Step 4.}
\[
t =\Bigg(
max= \left\{\frac{1}{1},\frac{2}{0},\frac{0}{5}\right\}
\Bigg)^{-1}\ =\ 1
\]
\subsection{Step 5.}
\[
In\ step\ 4, \ the\ ratio\ corresponds\ to\ basic\ index\ 3
\]
\[
i\ = \ 3
\]
\subsection{Step 6.}
\[
\Delta z_{\mathcal N}= -( B^{-1} N )^{T}e_j = -\
\begin{bmatrix}
1 & 2 & 0 \\
-1 & -1 & -1 \\
\end{bmatrix}
\begin{bmatrix}
1 \\
0\\
0\\
\end{bmatrix}
= \begin{bmatrix}
-1 \\
1 \\
\end{bmatrix}
\]


\subsection{Step 7.}
\[
s \ =\ \frac{z_{\mathcal N}^{*}}{ \Delta z_{\mathcal N}}\ =\ \frac{-4}{-1}\ =\ 4
\]

\subsection{Step 8.}
\[
x_{1}^{*}\ =\ 1, \quad x_{\mathcal B}^{*}\ =\
\begin{bmatrix}
1 \\
3 \\
5 \\
\end{bmatrix}\ -1\
\begin{bmatrix}
1 \\
2 \\
0 \\
\end{bmatrix}\ =\
\begin{bmatrix}
0 \\
1 \\
5 \\
\end{bmatrix}\ ,
\]
\\

\[
z_{3}^{*}\ =\ 1, \quad z_{\mathcal N}^{*}\ =\
\begin{bmatrix}
-4 \\
-3 \\
\end{bmatrix}\ -4\
\begin{bmatrix}
-1 \\
1 \\
\end{bmatrix}\ =\
\begin{bmatrix}
0 \\
-7 \\
\end{bmatrix}\ ,
\]

\subsection{Step 9.}

\[ New\ set\  of\  basic\  and\  nonbasic\  indices \]
\[
\beta= \left\{1,4,5\right\} \quad and \quad  \nu=\left\{3,2\right\}
\]

\[
Corresponding\ new\ basis\ and\ nonbasis\ submatrices\ of\ A,
\]
\[
B =
\begin{bmatrix}
1 & 0 & 0 \\
0 &1 & 0\\
0 & 0 & 1\\
\end{bmatrix} \quad and \quad
\mathit{N} =
\begin{bmatrix}
1 & -1 \\
2 &-1\\
0 &1\\
\end{bmatrix}
\]

\[
New\ Basic\ primal\ variables\ and\ nonbasic\ dual\ variables:
\]
\[
x_{\mathcal B}^{*}\ =\
\begin{bmatrix}
x_{1}^{*} \\
x_{4}^{*} \\
x_{5}^{*} \\
\end{bmatrix}\ =\
\begin{bmatrix}
1 \\
1 \\
5 \\
\end{bmatrix}\
\quad
z_{\mathcal N}^{*}\ =\
\begin{bmatrix}
z_{3}^{*} \\
z_{2}^{*} \\
\end{bmatrix}\ =\
\begin{bmatrix}
4 \\
-7 \\
\end{bmatrix}
\]

\[
Since\ z_{\mathcal N}^{*}\ has\ all\ non\ negative\ components,\ Optimal\ value\ is
\]
\[
\zeta^{*}\ =\ 4x_{1}^{*}\ +\ 3x_{2}^{*}\ =\ 31
\]


\end{document}